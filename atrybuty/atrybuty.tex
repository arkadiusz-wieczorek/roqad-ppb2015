\documentclass[a4paper,11pt]{article}
%\usepackage[T1]{fontenc}
\usepackage{polski}
\usepackage[utf8]{inputenc}
\usepackage{microtype}
\usepackage{color}
\usepackage{amsfonts}
\usepackage{amsthm, amsmath}
\usepackage{amssymb}

\date{ }
\usepackage[
pdftitle={Letter},
colorlinks=true,linkcolor=black,urlcolor=black,citecolor=black]{hyperref}
\urlstyle{same}

\usepackage{geometry}
\geometry{total={210mm,297mm},
left=25mm,right=25mm,%
bindingoffset=0mm, top=25mm,bottom=25mm}

\linespread{1.3}
\pagestyle{empty}


\begin{document}

\title{\textbf{Atrybuty}}

\maketitle

Dla każdych dwóch urządzeń porównujemy ich atrybuty i otrzymujemy ciąg podobieństwa, o wartościach z przedziału $[0,1]$. 
Oznaczmy przez $D_1,D_2$ porównywane urządzenia.

\section{{\color{red} Czas korzystania}}
	Reprezentujemy przez ciąg 24 - elementowy. Jeżeli w całym badanym okresie, w danej godzinie pojawiła się aktywność, to stawiamy 1, w p.p 0. Miarę ustalimy na podstawie analizy przykładów użytkowników z kilkoma urządzeniami i zależności występujących pomiedzy nimi. Czy podobieństwo czasu jest informacją pozytywną czy negatywną? W obu przypadkach można zastosować miarę  $\frac{min\{c_1,c_2\}}{max\{c_1,c_2\}}$. (Chwilowo ignorujemy błędy wynikające z różnic stref czasowych do czasu uzyskania odpowiednich informacji.)
	
\section{Dominujący kraj}
	Dla każdego urządzenia tworzymy ciąg $(k_1,k_2,c)$, gdzie $k_1,\ k_2-$ to pierwszy i drugi najczęściej występujący kraj (przy czym kraj drugi wybieramy tylko, gdy występuje w co najmniej 30\% zapytań, w przeciwnym przypadku $k_2=0$), $c$ - liczba wszystkich krajów. Za tę samą wartość na miejscach 1 i 2 przyznajemy odpowiednio wartości 0.5, 0.2. Liczbę krajów porównujemy następująco: $0.3\cdot\frac{min\{c_1,c_2\}}{max\{c_1,c_2\}}$, gdzie $c_1$ liczba krajów dla $D_1$, $c_2$ - dla $D_2$. 

\section{Dominujący region}
	Jak dla kraju.

\section{Dominujące IP}
	Dla każdej godziny porównujemy dominujące IP. Liczbę takich samych wyników dzielimy przez liczbę godzin, w których oba urządzenia były aktywne.

\section{Dominujący ISP}
	Jak przy IP.

\section{{\color{red}Typ użytkownika określony względem URL}}
	Z pliku requests usuwamy wszystkie informacje poza dev i url; grupujemy je po takim samym dev i klastrujemy devices za pomocą programu R, na podstawie podobieństwa zbiorów url przypisanych do jednego device (typ użytkownika). Informacje o przyporządkowaniu do danego klastra porównujemy dla dwóch urządzeń zero-jedynkowo.

\section{Liczba odwiedzonych stron w ciągu godziny} \label{ls}
	Dla urządzenia tworzymy parę $(Max,med)$, gdzie $Max$ - to maksimum, a $med$ - mediana liczby stron otwartych w ciągu godziny. Dla $D_1$ i $D_2$ mamy odpowiednio pary $(Max_1,med_1)$, $(Max_2, med_2)$. Niech $M=max\{Max_1,Max_2\}$, $m=min\{Max_1,Max_2\}$. Miara:
	\[ \frac{1}{(1+M - m)^2} \ .\]
	Podobnie porównujemy medianę.

\section{Liczba stron startowych}

Atrybut określony na podstawie pierwszego połączenia w ciągu dnia. Porównujemy podobnie jak w punkcie \ref{ls}, z tym że ustalamy dodatkowy parametr $days$ - liczba dni aktywności urządzenia. Jeżeli $days < 10$, to pomijamy porównywanie.


\section{{\color{red} Typ połączenia (connection type)}}
	Mamy 5 typów połączeń:
	\begin{enumerate}		
	\item The database identifies dial-up (modem telefoniczny)
	\item cellular (komórkowy) 
	\item cable
	\item DSL
	\item corporate connection speeds (wszyscy mają jedno IP)
\end{enumerate}
	Miara jak dla kraju.

\end{document}